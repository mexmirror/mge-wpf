\section{GUI Design}
\subsection{WPF Ressourcen}
Mit WPF Ressourcen kann man nützliche Objekte (Brushes, Styles, Templates etc.) zentral definieren und wiederverwenden.
\subsection{Resources}
\paragraph{Physikalische Ressourcen} In den File Properties kann man eine Datei als WPF-Ressource deklarieren welche dann in die Binärdatei einkompiliert werden. Zugegriffen kann dann mittels Resource-Key (relativer Pfadname).
\paragraph{Resource} Jedes beliebige Objekt in XAML kann als Ressource definiert werden. Es bird mit einem Key-Attribut und dem x-Namespace benannt.
\begin{lstlisting}[language=xml]
<Application.Resources>
    <SolidColorBrush x:Key ="MyButtonBackground" Color="#EEEEEE" />
</Application.Resources>
\end{lstlisting}
Unter diesem Key sind Ressourcen dann auch ansprechbar
\begin{lstlisting}[language=xml]
<Button Background="{StaticResource MyButtonBackground}" Content="Save" />
\end{lstlisting}
\paragraph{ResourceDictionary} Ist ein Behälter um Ressourcen zu speichern. Es indexiert nach dem Ressourcen-Namen (\verb+x:Key+) und kann in allen Elementen, welche von FrameworkElement ableiten, verwendet werden. Wenn auf eine Ressource zugegriffen werden möchte, wird \begin{enumerate*}[label=\itshape \arabic*\upshape.]\item Key im Element und allen Parent-Nodes gesucht (Logical Tree), \item dann wird der Key in \verb+Application.Resources+ gesucht, \item als letztes wird in System-Ressourcen gesucht.\end{enumerate*}. Die Reheinfolge im XAML-Code ist wichtig.
\paragraph{System Ressources} Auf Ressourcen in Namespace \verb+System.Windows+ kann man mittels statischer Properties zugegriffen werden. Dazu gehören: \verb+SystemColors+, \verb+SystemFonts+ und \verb+SystemParamters+.
\begin{lstlisting}[language=xml]
StaticResource="{StaticResource[name]}"
\end{lstlisting}
Die Statische Bindung macht CompileTime Check und findet Fehler früh.
\begin{lstlisting}[language=xml]
DynamicResource="{DynamicResource[name]}"
\end{lstlisting}
Anstelle von \verb+[name]+ steht der Key der Ressource
In der Dynamische Bindung wird ein RuntimeCheck gemacht und lässt dynamisch erzeugte und geladene Ressourcen zu. Beim Static Binding wird bei Objektkonstruktion ausgewertet, beim Dynamic Binding einmal pro Zugriff.
\paragraph{FindResource} Mit der Methode \verb+FrameworkElement.FindResources+  können Zugriffe auf XAML Resourcen gemacht werden.
\begin{lstlisting}
var okText = (string)FindResource("OkText");
var bgBrush = FindResource("DarkBrush") as Brush;
\end{lstlisting}